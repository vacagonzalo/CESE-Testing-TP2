\documentclass[
    11pt,
    spanish,
	a4paper
]{article}
\usepackage[utf8]{inputenc}
\usepackage[spanish]{babel}
\usepackage{graphicx}
\usepackage{authoraftertitle}
\usepackage{float}
\usepackage{caption}
\captionsetup[table]{labelformat=empty}

\def\doctype{PRUEBAS DE ACEPTACIÓN Y SISTEMA}
\title{Evaluador de microcontroladores para misiones espaciales}
\author{Gonzalo Nahuel Vaca}

\begin{document}

\makeatletter
\begin{titlepage}
	\begin{center}
		\vspace*{1cm}
		
		\Huge
		\textbf{\doctype}
		
		\vspace{0.5cm}
		\LARGE
		\@title
		
		\vspace{1.5cm}
		
		\textbf{\@author}

		\vspace{3.5cm}

		\includegraphics[width=0.8\textwidth]{img/logoFIUBA.pdf}
		
		\vfill
		Facultad de Ingeniería\\
		Universidad de Buenos Aires\\
		Argentina\\
		\today
	\end{center}
\end{titlepage}
\makeatother
\newpage

\section{Introducción}
\label{sec:introduccion}

Este documento tiene el objetivo de planificar los ensayos de nivel de sistema y de nivel de aceptación para un módulo del proyecto ``\MyTitle''.

El módulo seleccionado es el ``Generador de informe de secuencia'', quién se encarga reportar el estado del microcontrolador al ``Sistema de inyección de \emph{soft-errors}''.

\section{Pruebas a nivel sistemas}
\label{sec:lvlsistema}

\subsection{Requerimientos de software}
\label{sub:reqsoftware}

\begin{itemize}
    \item El reset del watchdog pondrá en 1 el bit menos significativo del byte de periféricos.
    \item Una falla en el periférico SPI pondrá en 1 el bit 2 del byte de periféricos.
    \item Se generará un byte de CRC al final de la trama utilizando el polinomio: 
$$ 0x^{7} + 0x^{6} + 0x^{5} + x^{4} + x^{3} + x^{2} + 0x^{1} + 0x $$
\end{itemize}

\section{Pruebas a nivel aceptación}
\label{sec:lvlaceptacion}

\subsection{Requerimientos de usuario}
\label{sub:requsuario}

\begin{itemize}
    \item Los reportes tendrán un número de 32 bits que lo identifique.
    \item Los reportes tendrán un byte que informe el estado de los periféricos.
\end{itemize}

\end{document}
